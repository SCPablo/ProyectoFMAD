% Options for packages loaded elsewhere
\PassOptionsToPackage{unicode}{hyperref}
\PassOptionsToPackage{hyphens}{url}
%
\documentclass[
]{article}
\usepackage{amsmath,amssymb}
\usepackage{lmodern}
\usepackage{setspace}
\usepackage{ifxetex,ifluatex}
\ifnum 0\ifxetex 1\fi\ifluatex 1\fi=0 % if pdftex
  \usepackage[T1]{fontenc}
  \usepackage[utf8]{inputenc}
  \usepackage{textcomp} % provide euro and other symbols
\else % if luatex or xetex
  \usepackage{unicode-math}
  \defaultfontfeatures{Scale=MatchLowercase}
  \defaultfontfeatures[\rmfamily]{Ligatures=TeX,Scale=1}
\fi
% Use upquote if available, for straight quotes in verbatim environments
\IfFileExists{upquote.sty}{\usepackage{upquote}}{}
\IfFileExists{microtype.sty}{% use microtype if available
  \usepackage[]{microtype}
  \UseMicrotypeSet[protrusion]{basicmath} % disable protrusion for tt fonts
}{}
\makeatletter
\@ifundefined{KOMAClassName}{% if non-KOMA class
  \IfFileExists{parskip.sty}{%
    \usepackage{parskip}
  }{% else
    \setlength{\parindent}{0pt}
    \setlength{\parskip}{6pt plus 2pt minus 1pt}}
}{% if KOMA class
  \KOMAoptions{parskip=half}}
\makeatother
\usepackage{xcolor}
\IfFileExists{xurl.sty}{\usepackage{xurl}}{} % add URL line breaks if available
\IfFileExists{bookmark.sty}{\usepackage{bookmark}}{\usepackage{hyperref}}
\hypersetup{
  pdftitle={Proyecto FMAD},
  pdfauthor={Álvaro Rodríguez González, Ignacio Perez-Cea, Pablo Sanz Caperote},
  hidelinks,
  pdfcreator={LaTeX via pandoc}}
\urlstyle{same} % disable monospaced font for URLs
\usepackage[margin=1in]{geometry}
\usepackage{color}
\usepackage{fancyvrb}
\newcommand{\VerbBar}{|}
\newcommand{\VERB}{\Verb[commandchars=\\\{\}]}
\DefineVerbatimEnvironment{Highlighting}{Verbatim}{commandchars=\\\{\}}
% Add ',fontsize=\small' for more characters per line
\usepackage{framed}
\definecolor{shadecolor}{RGB}{248,248,248}
\newenvironment{Shaded}{\begin{snugshade}}{\end{snugshade}}
\newcommand{\AlertTok}[1]{\textcolor[rgb]{0.94,0.16,0.16}{#1}}
\newcommand{\AnnotationTok}[1]{\textcolor[rgb]{0.56,0.35,0.01}{\textbf{\textit{#1}}}}
\newcommand{\AttributeTok}[1]{\textcolor[rgb]{0.77,0.63,0.00}{#1}}
\newcommand{\BaseNTok}[1]{\textcolor[rgb]{0.00,0.00,0.81}{#1}}
\newcommand{\BuiltInTok}[1]{#1}
\newcommand{\CharTok}[1]{\textcolor[rgb]{0.31,0.60,0.02}{#1}}
\newcommand{\CommentTok}[1]{\textcolor[rgb]{0.56,0.35,0.01}{\textit{#1}}}
\newcommand{\CommentVarTok}[1]{\textcolor[rgb]{0.56,0.35,0.01}{\textbf{\textit{#1}}}}
\newcommand{\ConstantTok}[1]{\textcolor[rgb]{0.00,0.00,0.00}{#1}}
\newcommand{\ControlFlowTok}[1]{\textcolor[rgb]{0.13,0.29,0.53}{\textbf{#1}}}
\newcommand{\DataTypeTok}[1]{\textcolor[rgb]{0.13,0.29,0.53}{#1}}
\newcommand{\DecValTok}[1]{\textcolor[rgb]{0.00,0.00,0.81}{#1}}
\newcommand{\DocumentationTok}[1]{\textcolor[rgb]{0.56,0.35,0.01}{\textbf{\textit{#1}}}}
\newcommand{\ErrorTok}[1]{\textcolor[rgb]{0.64,0.00,0.00}{\textbf{#1}}}
\newcommand{\ExtensionTok}[1]{#1}
\newcommand{\FloatTok}[1]{\textcolor[rgb]{0.00,0.00,0.81}{#1}}
\newcommand{\FunctionTok}[1]{\textcolor[rgb]{0.00,0.00,0.00}{#1}}
\newcommand{\ImportTok}[1]{#1}
\newcommand{\InformationTok}[1]{\textcolor[rgb]{0.56,0.35,0.01}{\textbf{\textit{#1}}}}
\newcommand{\KeywordTok}[1]{\textcolor[rgb]{0.13,0.29,0.53}{\textbf{#1}}}
\newcommand{\NormalTok}[1]{#1}
\newcommand{\OperatorTok}[1]{\textcolor[rgb]{0.81,0.36,0.00}{\textbf{#1}}}
\newcommand{\OtherTok}[1]{\textcolor[rgb]{0.56,0.35,0.01}{#1}}
\newcommand{\PreprocessorTok}[1]{\textcolor[rgb]{0.56,0.35,0.01}{\textit{#1}}}
\newcommand{\RegionMarkerTok}[1]{#1}
\newcommand{\SpecialCharTok}[1]{\textcolor[rgb]{0.00,0.00,0.00}{#1}}
\newcommand{\SpecialStringTok}[1]{\textcolor[rgb]{0.31,0.60,0.02}{#1}}
\newcommand{\StringTok}[1]{\textcolor[rgb]{0.31,0.60,0.02}{#1}}
\newcommand{\VariableTok}[1]{\textcolor[rgb]{0.00,0.00,0.00}{#1}}
\newcommand{\VerbatimStringTok}[1]{\textcolor[rgb]{0.31,0.60,0.02}{#1}}
\newcommand{\WarningTok}[1]{\textcolor[rgb]{0.56,0.35,0.01}{\textbf{\textit{#1}}}}
\usepackage{graphicx}
\makeatletter
\def\maxwidth{\ifdim\Gin@nat@width>\linewidth\linewidth\else\Gin@nat@width\fi}
\def\maxheight{\ifdim\Gin@nat@height>\textheight\textheight\else\Gin@nat@height\fi}
\makeatother
% Scale images if necessary, so that they will not overflow the page
% margins by default, and it is still possible to overwrite the defaults
% using explicit options in \includegraphics[width, height, ...]{}
\setkeys{Gin}{width=\maxwidth,height=\maxheight,keepaspectratio}
% Set default figure placement to htbp
\makeatletter
\def\fps@figure{htbp}
\makeatother
\setlength{\emergencystretch}{3em} % prevent overfull lines
\providecommand{\tightlist}{%
  \setlength{\itemsep}{0pt}\setlength{\parskip}{0pt}}
\setcounter{secnumdepth}{-\maxdimen} % remove section numbering
\usepackage[spanish]{babel}
\ifluatex
  \usepackage{selnolig}  % disable illegal ligatures
\fi

\title{Proyecto FMAD}
\usepackage{etoolbox}
\makeatletter
\providecommand{\subtitle}[1]{% add subtitle to \maketitle
  \apptocmd{\@title}{\par {\large #1 \par}}{}{}
}
\makeatother
\subtitle{ICAI. Máster en Big Data. Fundamentos Matemáticos del Análisis
de Datos (FMAD).}
\author{Álvaro Rodríguez González, Ignacio Perez-Cea, Pablo Sanz
Caperote}
\date{Curso 2021-22. Última actualización: 2021-11-20}

\begin{document}
\maketitle

\setstretch{1.25}
\newpage
\tableofcontents
\newpage

\hypertarget{introducciuxf3n}{%
\section{1. Introducción}\label{introducciuxf3n}}

El principal objetivo de este proyecto es plasmar los conocimientos
adquiridos durante la primera parte del curso en la asignatura de
Fundamentos Matemáticos del Análisis de Datos. Para ello trabajaremos
sobre un dataset de campañas de marketing en EEUU.

Nuestra idea es primero realizar un breve estudio de nuestras variables,
donde realizaremos cambios en caso de considerarlo oportuno (usando el
paquete tidyverse), este estudio será tanto gráfico como no gráfico.

Una vez que hemos realizado la limpieza de los datos, realizaremos un
análisis gráfico sobre el comportamiento de las variables así como
también sobre alguna posible relación que nos resulte interesante.

Para terminar el proyecto aplicaremos técnicas de Machine Learning que
hemos aprendido durante la segunda parte del cuatrimestre.

Antes de iniciar el breve estudio, lo que haremos será cargar las
diferentes librerias que usaremos para nuestro proyecto. Entre ellas
encontraremos librerias públicas como tidyverse y algunas privadas como
MLTools:

\begin{Shaded}
\begin{Highlighting}[]
\FunctionTok{library}\NormalTok{(tidyverse)}
\FunctionTok{library}\NormalTok{(lubridate)}
\FunctionTok{library}\NormalTok{(caret)}
\FunctionTok{library}\NormalTok{(grid)}
\FunctionTok{library}\NormalTok{(corrplot)}
\FunctionTok{library}\NormalTok{(gridExtra)}
\FunctionTok{library}\NormalTok{(ROCR)}
\FunctionTok{library}\NormalTok{(MLTools)}
\FunctionTok{library}\NormalTok{(GGally)}
\FunctionTok{library}\NormalTok{(rpart)}
\FunctionTok{library}\NormalTok{(rpart.plot)}
\FunctionTok{library}\NormalTok{(partykit)}
\FunctionTok{library}\NormalTok{(kernlab)}
\FunctionTok{library}\NormalTok{(NeuralNetTools) }
\FunctionTok{library}\NormalTok{(NeuralSens)}
\FunctionTok{library}\NormalTok{(nnet)}
\FunctionTok{library}\NormalTok{(ROSE)}
\FunctionTok{library}\NormalTok{(randomForest)}
\end{Highlighting}
\end{Shaded}

A su vez tambien leeremos los datos con los que trabajaremos:

\begin{Shaded}
\begin{Highlighting}[]
\NormalTok{datos }\OtherTok{\textless{}{-}} \FunctionTok{read.csv}\NormalTok{(}\StringTok{"marketing\_campaign.csv"}\NormalTok{, }\AttributeTok{header =} \ConstantTok{TRUE}\NormalTok{, }\AttributeTok{sep =} \StringTok{""}\NormalTok{)}
\end{Highlighting}
\end{Shaded}

\newpage

\hypertarget{definiciuxf3n-de-las-variables}{%
\section{2. Definición de las
variables}\label{definiciuxf3n-de-las-variables}}

Antes de comenzar con el preprocesamiento de los datos lo que haremos
será listar las variables y lo que representa cada una de ellas:

\begin{itemize}
\item
  \textbf{ID}: El ID del cliente.
\item
  \textbf{Year\_Birth:} Indica el año de nacimiento del cliente.
\item
  \textbf{Education:} Indica el nivel de educación del cliente.
\item
  \textbf{Marital\_Status:} Indica el estado civil del cliente.
\item
  \textbf{Income:} Presenta el ingreso familiar anual del cliente.
\item
  \textbf{Kidhome:} Indica el número de niños pequeños en casa del
  cliente.
\item
  \textbf{Teenhome:} Indica el número de adolescentes en el hogar del
  cliente.
\item
  \textbf{Dt\_Customer:} Muestra la fecha de inscripción del cliente en
  la empresa.
\item
  \textbf{Recency:} El número de días desde la última compra.
\item
  \textbf{MntWines:} El gasto en productos vitivinícolas en los últimos
  2 años.
\item
  \textbf{MntGoldProds:} El gasto en productos premium en los últimos 2
  años.
\item
  \textbf{NumDealsPurchases:} El número de compras con uso de descuento.
\item
  \textbf{NumWebPurchases:} El número de compras a través de la web.
\item
  \textbf{NumCatalogPurchases:} El número de compras usando catalogo.
\item
  \textbf{NumWebVisitsMonth:} El número de visitas por mes a la web.
\item
  \textbf{AcceptedCmp1:} 1 si el cliente acepta la oferta en la 1ra
  campaña, 0 si no lo acepta.
\item
  \textbf{AcceptedCmp2:} 1 si el cliente acepta la oferta en la 2nd
  campaña, 0 si no lo acepta.
\item
  \textbf{Complain:} 1 si el cliente se ha quejado en los dos últimos
  años.
\item
  \textbf{Z\_CostContact:} El coste de contactar con cliente.
\item
  \textbf{Z\_Revenue:} Los ingresos/beneficios después de que el cliente
  acepte la campaña.
\item
  \textbf{Response:} 1 si el cliente acepta la oferta en la última
  campaña y 0 si no la acepta.
\end{itemize}

\newpage

\hypertarget{preprocesamiento}{%
\section{3. Preprocesamiento}\label{preprocesamiento}}

\hypertarget{resumen-de-datos}{%
\subsection{3.1. Resumen de datos}\label{resumen-de-datos}}

Lo primero que haremos será ver como esta estructurado nuestro dataset.
Para ello veremos que tamaño tiene, tanto filas como columnas. A su vez
también veremos con que tipo de datos estamos trabajando.

\begin{Shaded}
\begin{Highlighting}[]
\FunctionTok{cat}\NormalTok{(}\FunctionTok{cat}\NormalTok{(}\FunctionTok{cat}\NormalTok{(}\FunctionTok{cat}\NormalTok{(}\StringTok{"El conjunto de datos tiene"}\NormalTok{, }\FunctionTok{nrow}\NormalTok{(datos)), }\StringTok{"filas y"}\NormalTok{), }
        \FunctionTok{ncol}\NormalTok{(datos)), }\StringTok{"columnas"}\NormalTok{)}
\end{Highlighting}
\end{Shaded}

\begin{verbatim}
## El conjunto de datos tiene 2440 filas y 29 columnas
\end{verbatim}

\begin{Shaded}
\begin{Highlighting}[]
\FunctionTok{str}\NormalTok{(datos)}
\end{Highlighting}
\end{Shaded}

\begin{verbatim}
## 'data.frame':    2440 obs. of  29 variables:
##  $ ID                 : int  5524 2174 4141 6182 5324 7446 965 6177 4855 5899 ...
##  $ Year_Birth         : int  1957 1954 1965 1984 1981 1967 1971 1985 1974 1950 ...
##  $ Education          : chr  "Graduation" "Graduation" "Graduation" "Graduation" ...
##  $ Marital_Status     : chr  "Single" "Single" "Together" "Together" ...
##  $ Income             : chr  "58138" "46344" "71613" "26646" ...
##  $ Kidhome            : int  0 1 0 1 1 0 0 1 1 1 ...
##  $ Teenhome           : chr  "0" "1" "0" "0" ...
##  $ Dt_Customer        : chr  "04-09-2012" "08-03-2014" "21-08-2013" "10-02-2014" ...
##  $ Recency            : chr  "58" "38" "26" "26" ...
##  $ MntWines           : int  635 11 426 11 173 520 235 76 14 28 ...
##  $ MntFruits          : int  88 1 49 4 43 42 65 10 0 0 ...
##  $ MntMeatProducts    : int  546 6 127 20 118 98 164 56 24 6 ...
##  $ MntFishProducts    : int  172 2 111 10 46 0 50 3 3 1 ...
##  $ MntSweetProducts   : int  88 1 21 3 27 42 49 1 3 1 ...
##  $ MntGoldProds       : int  88 6 42 5 15 14 27 23 2 13 ...
##  $ NumDealsPurchases  : int  3 2 1 2 5 2 4 2 1 1 ...
##  $ NumWebPurchases    : int  8 1 8 2 5 6 7 4 3 1 ...
##  $ NumCatalogPurchases: int  10 1 2 0 3 4 3 0 0 0 ...
##  $ NumStorePurchases  : int  4 2 10 4 6 10 7 4 2 0 ...
##  $ NumWebVisitsMonth  : int  7 5 4 6 5 6 6 8 9 20 ...
##  $ AcceptedCmp3       : int  0 0 0 0 0 0 0 0 0 1 ...
##  $ AcceptedCmp4       : int  0 0 0 0 0 0 0 0 0 0 ...
##  $ AcceptedCmp5       : int  0 0 0 0 0 0 0 0 0 0 ...
##  $ AcceptedCmp1       : int  0 0 0 0 0 0 0 0 0 0 ...
##  $ AcceptedCmp2       : int  0 0 0 0 0 0 0 0 0 0 ...
##  $ Complain           : int  0 0 0 0 0 0 0 0 0 0 ...
##  $ Z_CostContact      : int  3 3 3 3 3 3 3 3 3 3 ...
##  $ Z_Revenue          : int  11 11 11 11 11 11 11 11 11 11 ...
##  $ Response           : int  1 0 0 0 0 0 0 0 1 0 ...
\end{verbatim}

Una vez visto el tipo de variables con las que trabajamos es facilmente
observable la necesidad de realizar algunas modificaciones en algunas de
ellas.

Ahora veremos un resumen de las variables que tenemos:

\begin{Shaded}
\begin{Highlighting}[]
\FunctionTok{summary}\NormalTok{(datos)}
\end{Highlighting}
\end{Shaded}

\begin{verbatim}
##        ID          Year_Birth    Education         Marital_Status    
##  Min.   :    0   Min.   :1893   Length:2440        Length:2440       
##  1st Qu.: 2108   1st Qu.:1959   Class :character   Class :character  
##  Median : 5048   Median :1970   Mode  :character   Mode  :character  
##  Mean   : 5134   Mean   :1969                                        
##  3rd Qu.: 8147   3rd Qu.:1977                                        
##  Max.   :11191   Max.   :1996                                        
##                  NA's   :200                                         
##     Income             Kidhome        Teenhome         Dt_Customer       
##  Length:2440        Min.   :    0   Length:2440        Length:2440       
##  Class :character   1st Qu.:    0   Class :character   Class :character  
##  Mode  :character   Median :    0   Mode  :character   Mode  :character  
##                     Mean   : 4253                                        
##                     3rd Qu.:    1                                        
##                     Max.   :96547                                        
##                     NA's   :200                                          
##    Recency             MntWines        MntFruits       MntMeatProducts 
##  Length:2440        Min.   :   0.0   Min.   :   0.00   Min.   :   0.0  
##  Class :character   1st Qu.:  25.0   1st Qu.:   2.00   1st Qu.:  14.0  
##  Mode  :character   Median : 138.0   Median :   9.00   Median :  57.0  
##                     Mean   : 288.5   Mean   :  42.39   Mean   : 156.8  
##                     3rd Qu.: 476.5   3rd Qu.:  38.00   3rd Qu.: 211.5  
##                     Max.   :1493.0   Max.   :1215.00   Max.   :1725.0  
##                     NA's   :200      NA's   :200       NA's   :200     
##  MntFishProducts  MntSweetProducts  MntGoldProds    NumDealsPurchases
##  Min.   :  0.00   Min.   :  0.00   Min.   :  0.00   Min.   :  0.000  
##  1st Qu.:  3.00   1st Qu.:  1.00   1st Qu.:  7.00   1st Qu.:  1.000  
##  Median : 13.00   Median :  9.00   Median : 22.00   Median :  2.000  
##  Mean   : 45.34   Mean   : 28.44   Mean   : 42.45   Mean   :  6.326  
##  3rd Qu.: 55.00   3rd Qu.: 35.00   3rd Qu.: 54.00   3rd Qu.:  4.000  
##  Max.   :974.00   Max.   :362.00   Max.   :321.00   Max.   :246.000  
##  NA's   :200      NA's   :200      NA's   :200      NA's   :200      
##  NumWebPurchases  NumCatalogPurchases NumStorePurchases NumWebVisitsMonth
##  Min.   : 0.000   Min.   : 0.000      Min.   : 0.000    Min.   : 0.000   
##  1st Qu.: 2.000   1st Qu.: 1.000      1st Qu.: 3.000    1st Qu.: 3.000   
##  Median : 3.000   Median : 2.000      Median : 5.000    Median : 6.000   
##  Mean   : 3.929   Mean   : 2.817      Mean   : 5.503    Mean   : 5.278   
##  3rd Qu.: 6.000   3rd Qu.: 4.000      3rd Qu.: 8.000    3rd Qu.: 7.000   
##  Max.   :27.000   Max.   :28.000      Max.   :13.000    Max.   :20.000   
##  NA's   :200      NA's   :200         NA's   :200       NA's   :200      
##   AcceptedCmp3     AcceptedCmp4      AcceptedCmp5      AcceptedCmp1    
##  Min.   :0.0000   Min.   :0.00000   Min.   :0.00000   Min.   :0.00000  
##  1st Qu.:0.0000   1st Qu.:0.00000   1st Qu.:0.00000   1st Qu.:0.00000  
##  Median :0.0000   Median :0.00000   Median :0.00000   Median :0.00000  
##  Mean   :0.5545   Mean   :0.07679   Mean   :0.07277   Mean   :0.06161  
##  3rd Qu.:0.0000   3rd Qu.:0.00000   3rd Qu.:0.00000   3rd Qu.:0.00000  
##  Max.   :9.0000   Max.   :1.00000   Max.   :1.00000   Max.   :1.00000  
##  NA's   :200      NA's   :200       NA's   :200       NA's   :200      
##   AcceptedCmp2        Complain       Z_CostContact      Z_Revenue    
##  Min.   :0.00000   Min.   :0.00000   Min.   : 0.000   Min.   : 0.00  
##  1st Qu.:0.00000   1st Qu.:0.00000   1st Qu.: 3.000   1st Qu.:11.00  
##  Median :0.00000   Median :0.00000   Median : 3.000   Median :11.00  
##  Mean   :0.01875   Mean   :0.03661   Mean   : 2.809   Mean   :10.18  
##  3rd Qu.:0.00000   3rd Qu.:0.00000   3rd Qu.: 3.000   3rd Qu.:11.00  
##  Max.   :1.00000   Max.   :3.00000   Max.   :11.000   Max.   :11.00  
##  NA's   :200       NA's   :200       NA's   :200      NA's   :200    
##     Response     
##  Min.   : 0.000  
##  1st Qu.: 0.000  
##  Median : 0.000  
##  Mean   : 1.132  
##  3rd Qu.: 0.000  
##  Max.   :11.000  
##  NA's   :221
\end{verbatim}

Observamos que existen numerosos valores nulos en nuestras variables. En
el punto siguiente veremos que hacer con estos casos.

\hypertarget{anuxe1lisis-de-las-variables}{%
\subsection{3.2. Análisis de las
variables}\label{anuxe1lisis-de-las-variables}}

Lo primero que haremos será eliminar las filas que contienen datos
nulos. Esto podemos hacerlo ya que disponemos de una muestra muy grande
y eliminar los valores nulos no afectará para nuestro trabajo.

\begin{Shaded}
\begin{Highlighting}[]
\NormalTok{datos }\OtherTok{\textless{}{-}} \FunctionTok{na.omit}\NormalTok{(datos)}
\end{Highlighting}
\end{Shaded}

Además, también observamos que hay algunos datos erróneos por lo que por
el mismo motivo que antes procederemos a eliminarlos.

\begin{Shaded}
\begin{Highlighting}[]
\NormalTok{datos }\OtherTok{\textless{}{-}}\NormalTok{ datos }\SpecialCharTok{\%\textgreater{}\%}
  \FunctionTok{filter}\NormalTok{(ID }\SpecialCharTok{!=} \DecValTok{0} \SpecialCharTok{\&}\NormalTok{ ID }\SpecialCharTok{!=} \DecValTok{1} \SpecialCharTok{\&}\NormalTok{ Education }\SpecialCharTok{!=} \StringTok{"2n"} \SpecialCharTok{\&}\NormalTok{ Income }\SpecialCharTok{\textgreater{}} \DecValTok{10} \SpecialCharTok{\&}\NormalTok{ Income }\SpecialCharTok{!=} \StringTok{"2"}\NormalTok{)}
\end{Highlighting}
\end{Shaded}

Además el conjunto de datos tiene muchas columnas las cuales no nos
resultan interesantes, por ello vamos a eliminar algunas de ellas:
``NumDealsPurchases'', ``Receny'', ``AcceptedCmp1'', ``AcceptedCmp2'',
``AcceptedCmp3'', ``AcceptedCmp4'' y ``AcceptedCmp5'',
``Z\_CostContact'' y ``Z\_Revenue''.

\begin{Shaded}
\begin{Highlighting}[]
\NormalTok{datos }\OtherTok{\textless{}{-}}\NormalTok{ datos }\SpecialCharTok{\%\textgreater{}\%}
  \FunctionTok{select}\NormalTok{(}\SpecialCharTok{{-}}\FunctionTok{c}\NormalTok{(AcceptedCmp3}\SpecialCharTok{:}\NormalTok{AcceptedCmp2), }\SpecialCharTok{{-}}\NormalTok{NumDealsPurchases, }\SpecialCharTok{{-}}\NormalTok{Recency, }
         \SpecialCharTok{{-}}\NormalTok{Z\_CostContact, }\SpecialCharTok{{-}}\NormalTok{Z\_Revenue)}
\end{Highlighting}
\end{Shaded}

También como vimos cuando hicimos la visión general de las variables y
su tipo, nos dimos cuenta de que algunas de ellas estaban mal tipadas.
Por ello cambiaremos el tipado de algunas columnas.

\begin{Shaded}
\begin{Highlighting}[]
\NormalTok{datos}\SpecialCharTok{$}\NormalTok{Teenhome }\OtherTok{\textless{}{-}} \FunctionTok{as.numeric}\NormalTok{(datos}\SpecialCharTok{$}\NormalTok{Teenhome)}
\NormalTok{datos}\SpecialCharTok{$}\NormalTok{Income }\OtherTok{\textless{}{-}} \FunctionTok{as.numeric}\NormalTok{(datos}\SpecialCharTok{$}\NormalTok{Income)}
\NormalTok{datos}\SpecialCharTok{$}\NormalTok{Complain }\OtherTok{\textless{}{-}} \FunctionTok{as.factor}\NormalTok{(datos}\SpecialCharTok{$}\NormalTok{Complain)}
\NormalTok{datos}\SpecialCharTok{$}\NormalTok{Education }\OtherTok{\textless{}{-}} \FunctionTok{as.factor}\NormalTok{(datos}\SpecialCharTok{$}\NormalTok{Education)}
\NormalTok{datos}\SpecialCharTok{$}\NormalTok{Response }\OtherTok{\textless{}{-}} \FunctionTok{as.factor}\NormalTok{(datos}\SpecialCharTok{$}\NormalTok{Response)}
\end{Highlighting}
\end{Shaded}

A su vez hemos observado que algunas columnas podrían tener un formato
más útil o sencillo, como es el caso del año de nacimiento, donde es mas
cómodo trabajar con edades. Por tanto, para un mejor procesamiento y una
mayor útilidad realizaremos un mutate para generar una nueva columna
formada por la edad de los clientes. A su vez eliminaremos la columna de
año de nacimiento.

\begin{Shaded}
\begin{Highlighting}[]
\NormalTok{datos }\OtherTok{\textless{}{-}}\NormalTok{ datos }\SpecialCharTok{\%\textgreater{}\%}
  \FunctionTok{mutate}\NormalTok{(}\AttributeTok{edad =} \DecValTok{2021} \SpecialCharTok{{-}}\NormalTok{ Year\_Birth) }\SpecialCharTok{\%\textgreater{}\%}
  \FunctionTok{select}\NormalTok{(}\SpecialCharTok{{-}}\NormalTok{Year\_Birth) }
\end{Highlighting}
\end{Shaded}

También nos pareció interesante en vez de distinguir entre número de
hijos los cuales son pequeños o son adolescentes, tomarlos como una
única variable que nos indique el número de hijos que hay en cada hogar.
Para ello sumaremos el total de niños de cada cliente agrupando las
columnas Kidhome y Teenhome.

\begin{Shaded}
\begin{Highlighting}[]
\NormalTok{datos }\OtherTok{\textless{}{-}}\NormalTok{ datos }\SpecialCharTok{\%\textgreater{}\%}
  \FunctionTok{mutate}\NormalTok{(}\AttributeTok{totalHijos =}\NormalTok{ Kidhome }\SpecialCharTok{+}\NormalTok{ Teenhome) }\SpecialCharTok{\%\textgreater{}\%}
  \FunctionTok{select}\NormalTok{(}\SpecialCharTok{{-}}\NormalTok{Kidhome, }\SpecialCharTok{{-}}\NormalTok{Teenhome)}
\end{Highlighting}
\end{Shaded}

Como en el caso de los hijos para las compras haremos algo similar,
donde cogeremos las columnas ``NumWebPurchases'',
``NumCatalogPurchases'' y ``NumStorePurchases'' que indican el número de
compras hechas en cada sitio, en tiendas, por catalogo y por la web y
las sumaremos todas en una única columna que indique el total de compras
que ha realizado el cliente.

\begin{Shaded}
\begin{Highlighting}[]
\NormalTok{datos }\OtherTok{\textless{}{-}}\NormalTok{ datos }\SpecialCharTok{\%\textgreater{}\%}
  \FunctionTok{rowwise}\NormalTok{(ID) }\SpecialCharTok{\%\textgreater{}\%}
  \FunctionTok{mutate}\NormalTok{(}\AttributeTok{suma\_compras =} \FunctionTok{sum}\NormalTok{(}\FunctionTok{c}\NormalTok{(NumWebPurchases, NumCatalogPurchases, }
\NormalTok{                              NumStorePurchases))) }\SpecialCharTok{\%\textgreater{}\%}
  \FunctionTok{select}\NormalTok{(}\SpecialCharTok{{-}}\FunctionTok{c}\NormalTok{(NumWebPurchases, NumCatalogPurchases, NumStorePurchases))}
\end{Highlighting}
\end{Shaded}

A su vez, para el gasto en los diferentes tipos de producto sumaremos
las columans: ``MntWines'', ``MntFruits'', ``MntMeatProducts'',
``MntFishProducts'', ``MntSweetProducts'', ``MntGoldProds'' lo cual nos
indicara cuánto dinero se ha gastado un cliente en total.

\begin{Shaded}
\begin{Highlighting}[]
\NormalTok{datos }\OtherTok{\textless{}{-}}\NormalTok{ datos }\SpecialCharTok{\%\textgreater{}\%}
  \FunctionTok{rowwise}\NormalTok{(ID) }\SpecialCharTok{\%\textgreater{}\%}
  \FunctionTok{mutate}\NormalTok{(}\AttributeTok{Dinero\_Gastado =} \FunctionTok{sum}\NormalTok{(}\FunctionTok{c}\NormalTok{(MntWines, MntFruits, MntMeatProducts, MntFishProducts, }
\NormalTok{                                MntSweetProducts, MntGoldProds)))}
\end{Highlighting}
\end{Shaded}

También existe una variable que nos indica el estado civil del cliente,
al existir numerosas situaciones nosotros agruparemos el estado civil de
cada cliente y lo simplificamos para ver si vive solo o en pareja. Ya
que esto nos podrá resultar interesante para análisis posteriores.

\begin{Shaded}
\begin{Highlighting}[]
\NormalTok{datos }\OtherTok{\textless{}{-}}\NormalTok{ datos }\SpecialCharTok{\%\textgreater{}\%}
  \FunctionTok{mutate}\NormalTok{(}\AttributeTok{Marital\_Status =} \FunctionTok{factor}\NormalTok{(Marital\_Status}\SpecialCharTok{==} \StringTok{"Single"} \SpecialCharTok{|}\NormalTok{ Marital\_Status}\SpecialCharTok{==} \StringTok{"Divorced"}\NormalTok{, }
                          \AttributeTok{levels =} \FunctionTok{c}\NormalTok{(}\ConstantTok{TRUE}\NormalTok{, }\ConstantTok{FALSE}\NormalTok{), }\AttributeTok{labels =} \FunctionTok{c}\NormalTok{(}\StringTok{\textquotesingle{}Single\textquotesingle{}}\NormalTok{,}\StringTok{\textquotesingle{}Not single\textquotesingle{}}\NormalTok{)))}
\end{Highlighting}
\end{Shaded}

Por último cambiaremos la columna Dt\_Customer y estableceremos 3 grupos
que representan la longevidad del cliente en la empresa.

\begin{Shaded}
\begin{Highlighting}[]
\NormalTok{datos}\SpecialCharTok{$}\NormalTok{Dt\_Customer }\OtherTok{=} \FunctionTok{as.Date}\NormalTok{(datos}\SpecialCharTok{$}\NormalTok{Dt\_Customer, }\StringTok{"\%d{-}\%m{-}\%Y"}\NormalTok{)}
\NormalTok{fechaMinima }\OtherTok{=} \FunctionTok{min}\NormalTok{(datos}\SpecialCharTok{$}\NormalTok{Dt\_Customer)}
\NormalTok{datos}\SpecialCharTok{$}\NormalTok{Dt\_Customer }\OtherTok{\textless{}{-}}  \FunctionTok{factor}\NormalTok{(}\FunctionTok{cut\_number}\NormalTok{(}\FunctionTok{as.duration}\NormalTok{(datos}\SpecialCharTok{$}\NormalTok{Dt\_Customer}\SpecialCharTok{{-}}\NormalTok{fechaMinima), }
                      \AttributeTok{n =} \DecValTok{3}\NormalTok{), }\AttributeTok{labels =} \FunctionTok{c}\NormalTok{(}\StringTok{"Nuevo"}\NormalTok{, }\StringTok{"Con Experiencia"}\NormalTok{, }\StringTok{"Muy Antiguo"}\NormalTok{), }
                      \AttributeTok{ordered =} \ConstantTok{TRUE}\NormalTok{)}
\end{Highlighting}
\end{Shaded}

\hypertarget{visualizaciuxf3n-de-los-datos}{%
\subsection{3.3. Visualización de los
datos}\label{visualizaciuxf3n-de-los-datos}}

En este apartado lo que realizaremos será un análisis mediante gráficas
de las variables según su tipo, viendo si siguen una distribución normal
y si presentarían outliers entre otros factores.

Comenzaremos con las variables continuas, que son las siguientes:

\begin{Shaded}
\begin{Highlighting}[]
\NormalTok{h1 }\OtherTok{\textless{}{-}} \FunctionTok{ggplot}\NormalTok{(datos)}\SpecialCharTok{+}
  \FunctionTok{geom\_histogram}\NormalTok{(}\FunctionTok{aes}\NormalTok{(}\AttributeTok{x =}\NormalTok{ Income), }\AttributeTok{color =} \StringTok{"black"}\NormalTok{, }\AttributeTok{alpha =} \FloatTok{0.35}\NormalTok{)}
\NormalTok{h2 }\OtherTok{\textless{}{-}} \FunctionTok{ggplot}\NormalTok{(datos)}\SpecialCharTok{+}
  \FunctionTok{geom\_histogram}\NormalTok{(}\FunctionTok{aes}\NormalTok{(}\AttributeTok{x =}\NormalTok{ MntWines ), }\AttributeTok{color =} \StringTok{"black"}\NormalTok{, }\AttributeTok{alpha =} \FloatTok{0.35}\NormalTok{)}
\NormalTok{h3 }\OtherTok{\textless{}{-}} \FunctionTok{ggplot}\NormalTok{(datos)}\SpecialCharTok{+}
  \FunctionTok{geom\_histogram}\NormalTok{(}\FunctionTok{aes}\NormalTok{(}\AttributeTok{x =}\NormalTok{ MntFruits ), }\AttributeTok{color =} \StringTok{"black"}\NormalTok{, }\AttributeTok{alpha =} \FloatTok{0.35}\NormalTok{)}
\NormalTok{h4 }\OtherTok{\textless{}{-}} \FunctionTok{ggplot}\NormalTok{(datos)}\SpecialCharTok{+}
  \FunctionTok{geom\_histogram}\NormalTok{(}\FunctionTok{aes}\NormalTok{(}\AttributeTok{x =}\NormalTok{ MntMeatProducts ), }\AttributeTok{color =} \StringTok{"black"}\NormalTok{, }\AttributeTok{alpha =} \FloatTok{0.35}\NormalTok{)}
\NormalTok{h5 }\OtherTok{\textless{}{-}} \FunctionTok{ggplot}\NormalTok{(datos)}\SpecialCharTok{+}
  \FunctionTok{geom\_histogram}\NormalTok{(}\FunctionTok{aes}\NormalTok{(}\AttributeTok{x =}\NormalTok{ MntFishProducts ), }\AttributeTok{color =} \StringTok{"black"}\NormalTok{, }\AttributeTok{alpha =} \FloatTok{0.35}\NormalTok{)}
\NormalTok{h6 }\OtherTok{\textless{}{-}} \FunctionTok{ggplot}\NormalTok{(datos)}\SpecialCharTok{+}
  \FunctionTok{geom\_histogram}\NormalTok{(}\FunctionTok{aes}\NormalTok{(}\AttributeTok{x =}\NormalTok{ MntSweetProducts ), }\AttributeTok{color =} \StringTok{"black"}\NormalTok{, }\AttributeTok{alpha =} \FloatTok{0.35}\NormalTok{)}
\NormalTok{h7 }\OtherTok{\textless{}{-}} \FunctionTok{ggplot}\NormalTok{(datos) }\SpecialCharTok{+}
  \FunctionTok{geom\_histogram}\NormalTok{(}\FunctionTok{aes}\NormalTok{(}\AttributeTok{x =}\NormalTok{ MntGoldProds ), }\AttributeTok{color =} \StringTok{"black"}\NormalTok{, }\AttributeTok{alpha =} \FloatTok{0.35}\NormalTok{)}
\NormalTok{h8 }\OtherTok{\textless{}{-}} \FunctionTok{ggplot}\NormalTok{(datos) }\SpecialCharTok{+}
  \FunctionTok{geom\_histogram}\NormalTok{(}\FunctionTok{aes}\NormalTok{(}\AttributeTok{x =}\NormalTok{ Dinero\_Gastado), }\AttributeTok{color =} \StringTok{"black"}\NormalTok{, }\AttributeTok{alpha =} \FloatTok{0.35}\NormalTok{)}
\NormalTok{h9 }\OtherTok{\textless{}{-}} \FunctionTok{ggplot}\NormalTok{(datos)}\SpecialCharTok{+}
  \FunctionTok{geom\_histogram}\NormalTok{(}\FunctionTok{aes}\NormalTok{(}\AttributeTok{x =}\NormalTok{ suma\_compras ), }\AttributeTok{color =} \StringTok{"black"}\NormalTok{, }\AttributeTok{alpha =} \FloatTok{0.35}\NormalTok{)}
\NormalTok{h10 }\OtherTok{\textless{}{-}} \FunctionTok{ggplot}\NormalTok{(datos)}\SpecialCharTok{+}
  \FunctionTok{geom\_histogram}\NormalTok{(}\FunctionTok{aes}\NormalTok{(}\AttributeTok{x =}\NormalTok{ NumWebVisitsMonth ), }\AttributeTok{color =} \StringTok{"black"}\NormalTok{, }\AttributeTok{alpha =} \FloatTok{0.35}\NormalTok{)}
\NormalTok{h12 }\OtherTok{\textless{}{-}} \FunctionTok{ggplot}\NormalTok{(datos)}\SpecialCharTok{+}
  \FunctionTok{geom\_histogram}\NormalTok{(}\FunctionTok{aes}\NormalTok{(}\AttributeTok{x =}\NormalTok{ edad ), }\AttributeTok{color =} \StringTok{"black"}\NormalTok{, }\AttributeTok{alpha =} \FloatTok{0.35}\NormalTok{)}
\FunctionTok{grid.arrange}\NormalTok{(h1,h2,h3,h4,h5,h6,h7,h8,h9,h10,h12)}
\end{Highlighting}
\end{Shaded}

\begin{figure}
\centering
\includegraphics{Proyecto_files/figure-latex/unnamed-chunk-15-1.pdf}
\caption{Gráfico variables continuas}
\end{figure}

Podemos observar en los histogramas que ninguna de las variables sigue
una distribución normal. El patrón más común es un gran número de datos
con valores pequeños y muchos menos datos a medida que el valor de la
variable del eje x aumenta. Por este motivo, aunque no los mostremos,
podemos deducir que existen outliers en casi todas las variables. Un
caso bastante claro de outlier se puede ver en la variable edad donde
vemos que el eje x llega a 125 lo cual nos indica que debe existir
alguna observación con valor por encima de 115 y menor de 125.

En cuanto a las variables discretas tenemos lo siguiente:

\begin{Shaded}
\begin{Highlighting}[]
\NormalTok{hb1 }\OtherTok{\textless{}{-}} \FunctionTok{ggplot}\NormalTok{(datos)}\SpecialCharTok{+}
  \FunctionTok{geom\_bar}\NormalTok{(}\FunctionTok{aes}\NormalTok{(}\AttributeTok{x =}\NormalTok{ Education), }\AttributeTok{color =} \StringTok{"black"}\NormalTok{, }\AttributeTok{alpha =} \FloatTok{0.35}\NormalTok{)}
\NormalTok{hb2 }\OtherTok{\textless{}{-}} \FunctionTok{ggplot}\NormalTok{(datos) }\SpecialCharTok{+} 
  \FunctionTok{geom\_bar}\NormalTok{(}\FunctionTok{aes}\NormalTok{(}\AttributeTok{x =}\NormalTok{ Marital\_Status), }\AttributeTok{color =} \StringTok{"black"}\NormalTok{, }\AttributeTok{alpha =} \FloatTok{0.35}\NormalTok{)}
\NormalTok{hb3 }\OtherTok{\textless{}{-}} \FunctionTok{ggplot}\NormalTok{(datos) }\SpecialCharTok{+} 
  \FunctionTok{geom\_bar}\NormalTok{(}\FunctionTok{aes}\NormalTok{(}\AttributeTok{x =}\NormalTok{ Dt\_Customer), }\AttributeTok{color =} \StringTok{"black"}\NormalTok{, }\AttributeTok{alpha =} \FloatTok{0.35}\NormalTok{)}
\NormalTok{hb4 }\OtherTok{\textless{}{-}} \FunctionTok{ggplot}\NormalTok{(datos) }\SpecialCharTok{+} 
  \FunctionTok{geom\_bar}\NormalTok{(}\FunctionTok{aes}\NormalTok{(}\AttributeTok{x =}\NormalTok{ Complain), }\AttributeTok{color =} \StringTok{"black"}\NormalTok{, }\AttributeTok{alpha =} \FloatTok{0.35}\NormalTok{)}
\FunctionTok{grid.arrange}\NormalTok{(hb1, hb2, hb3, hb4, }\AttributeTok{ncol =} \DecValTok{2}\NormalTok{)}
\end{Highlighting}
\end{Shaded}

\begin{figure}

{\centering \includegraphics{Proyecto_files/figure-latex/unnamed-chunk-16-1} 

}

\caption{Gráfico variables discretas}\label{fig:unnamed-chunk-16}
\end{figure}

En estas variables podemos observar diferentes patrones. Si miramos la
educación de los clientes nos damos cuenta que hay mucha diferencia
entre el número de clientes que tienen una educación básica y el resto
de tipos, sobre todo los clientes graduados. De la misma forma vemos que
hay más del doble de clientes not single que single. Por otro lado
tenemos que la longividad de los clientes es uniforme. En cuanto a la
variable complain observamos que unicamente un porcentaje muy pequeño de
los clientes se han quejado durante los dos últimos años.

\newpage

\hypertarget{anuxe1lisis-gruxe1fico}{%
\section{4. Análisis Gráfico}\label{anuxe1lisis-gruxe1fico}}

En esta sección nuestro objetivo será sacar conclusiones acerca de
diversas variables así como de posibles relaciones entre estas. Nuestros
principales apoyos para sacar dichas conclusiones serán los elementos
gráficos (histogramas, boxplot, tablas, etc).

\hypertarget{relaciuxf3n-nuxba-de-hijos-gasto-y-estudios.}{%
\subsection{4.1. Relación nº de hijos, gasto y
estudios.}\label{relaciuxf3n-nuxba-de-hijos-gasto-y-estudios.}}

Para ello crearemos un nuevo conjunto de datos sacados de los datos que
hemos refinado en la parte anterior. A su vez en un primer momento hemos
analizado la relación entre tener hijos y los gastos en compras de cada
tipo.

\begin{Shaded}
\begin{Highlighting}[]
\NormalTok{datos\_ML}\OtherTok{\textless{}{-}}\NormalTok{datos }\SpecialCharTok{\%\textgreater{}\%}
  \FunctionTok{mutate}\NormalTok{(}\AttributeTok{sonPadres =} \FunctionTok{factor}\NormalTok{(totalHijos}\SpecialCharTok{\textgreater{}}\DecValTok{0}\NormalTok{, }\AttributeTok{levels =} \FunctionTok{c}\NormalTok{(}\ConstantTok{FALSE}\NormalTok{, }\ConstantTok{TRUE}\NormalTok{), }\AttributeTok{labels =} \FunctionTok{c}\NormalTok{(}\DecValTok{0}\NormalTok{,}\DecValTok{1}\NormalTok{)))}
\NormalTok{borrar }\OtherTok{\textless{}{-}} \FunctionTok{c}\NormalTok{(}\StringTok{"MntWines"}\NormalTok{, }\StringTok{"MntFruits"}\NormalTok{, }\StringTok{"MntMeatProducts"}\NormalTok{,}
            \StringTok{"MntFishProducts"}\NormalTok{, }\StringTok{"MntSweetProducts"}\NormalTok{,}
            \StringTok{"MntGoldProds"}\NormalTok{,}\StringTok{"Dinero\_Gastado"}\NormalTok{, }\StringTok{"sonPadres"}\NormalTok{, }\StringTok{"totalHijos"}\NormalTok{)}
\NormalTok{datos\_ML }\OtherTok{\textless{}{-}}\NormalTok{ datos\_ML[(}\FunctionTok{names}\NormalTok{(datos\_ML) }\SpecialCharTok{\%in\%}\NormalTok{ borrar)]}
\FunctionTok{head}\NormalTok{(datos\_ML) }
\end{Highlighting}
\end{Shaded}

\begin{verbatim}
## # A tibble: 6 x 9
## # Rowwise: 
##   MntWines MntFruits MntMeatProducts MntFishProducts MntSweetProducts
##      <int>     <int>           <int>           <int>            <int>
## 1      635        88             546             172               88
## 2       11         1               6               2                1
## 3      426        49             127             111               21
## 4       11         4              20              10                3
## 5      173        43             118              46               27
## 6      520        42              98               0               42
## # ... with 4 more variables: MntGoldProds <int>, totalHijos <dbl>,
## #   Dinero_Gastado <int>, sonPadres <fct>
\end{verbatim}

Lo primero que haremos sera ver como se relacionan las diferentes
variables con las variables de output, empezando por si tienen hijos.

\begin{Shaded}
\begin{Highlighting}[]
\NormalTok{g1 }\OtherTok{\textless{}{-}}\FunctionTok{ggplot}\NormalTok{(datos\_ML) }\SpecialCharTok{+} 
  \FunctionTok{geom\_boxplot}\NormalTok{(}\FunctionTok{aes}\NormalTok{(}\AttributeTok{x =}\NormalTok{ sonPadres, }\AttributeTok{y =}\NormalTok{ MntWines))}
\NormalTok{g2}\OtherTok{\textless{}{-}}\FunctionTok{ggplot}\NormalTok{(datos\_ML) }\SpecialCharTok{+} 
  \FunctionTok{geom\_boxplot}\NormalTok{(}\FunctionTok{aes}\NormalTok{(}\AttributeTok{x =}\NormalTok{ sonPadres, }\AttributeTok{y =}\NormalTok{ MntFruits))}
\NormalTok{g3}\OtherTok{\textless{}{-}}\FunctionTok{ggplot}\NormalTok{(datos\_ML) }\SpecialCharTok{+} 
  \FunctionTok{geom\_boxplot}\NormalTok{(}\FunctionTok{aes}\NormalTok{(}\AttributeTok{x =}\NormalTok{ sonPadres, }\AttributeTok{y =}\NormalTok{ MntMeatProducts))}
\NormalTok{g4}\OtherTok{\textless{}{-}}\FunctionTok{ggplot}\NormalTok{(datos\_ML) }\SpecialCharTok{+} 
  \FunctionTok{geom\_boxplot}\NormalTok{(}\FunctionTok{aes}\NormalTok{(}\AttributeTok{x =}\NormalTok{ sonPadres, }\AttributeTok{y =}\NormalTok{ MntFishProducts))}
\NormalTok{g5}\OtherTok{\textless{}{-}}\FunctionTok{ggplot}\NormalTok{(datos\_ML) }\SpecialCharTok{+} 
  \FunctionTok{geom\_boxplot}\NormalTok{(}\FunctionTok{aes}\NormalTok{(}\AttributeTok{x =}\NormalTok{ sonPadres, }\AttributeTok{y =}\NormalTok{ MntSweetProducts))}
\NormalTok{g6}\OtherTok{\textless{}{-}}\FunctionTok{ggplot}\NormalTok{(datos\_ML) }\SpecialCharTok{+} 
  \FunctionTok{geom\_boxplot}\NormalTok{(}\FunctionTok{aes}\NormalTok{(}\AttributeTok{x =}\NormalTok{ sonPadres, }\AttributeTok{y =}\NormalTok{ MntGoldProds))}
\NormalTok{g7}\OtherTok{\textless{}{-}}\FunctionTok{ggplot}\NormalTok{(datos\_ML) }\SpecialCharTok{+} 
  \FunctionTok{geom\_boxplot}\NormalTok{(}\FunctionTok{aes}\NormalTok{(}\AttributeTok{x =}\NormalTok{ sonPadres, }\AttributeTok{y =}\NormalTok{ Dinero\_Gastado))}
\NormalTok{gridExtra}\SpecialCharTok{::}\FunctionTok{grid.arrange}\NormalTok{(g1,g2,g3,g4, g5, g6, g7)}
\end{Highlighting}
\end{Shaded}

\begin{figure}
\centering
\includegraphics{Proyecto_files/figure-latex/unnamed-chunk-18-1.pdf}
\caption{Gráfico relaciones variables}
\end{figure}

Vemos que si los clientes tienen hijos, el gasto en todos los productos
se reduce. Pero ya que hemos llegado hasta aquí, queremos además
observar la relación entre el número de hijos y dichos gastos. Veamos
los siguientes gráficos.

\begin{Shaded}
\begin{Highlighting}[]
\NormalTok{gb1 }\OtherTok{\textless{}{-}}\FunctionTok{ggplot}\NormalTok{(datos\_ML) }\SpecialCharTok{+} 
  \FunctionTok{geom\_boxplot}\NormalTok{(}\FunctionTok{aes}\NormalTok{(}\AttributeTok{x =} \FunctionTok{factor}\NormalTok{(totalHijos), }\AttributeTok{y =}\NormalTok{ MntWines))}
\NormalTok{gb2}\OtherTok{\textless{}{-}}\FunctionTok{ggplot}\NormalTok{(datos\_ML) }\SpecialCharTok{+} 
  \FunctionTok{geom\_boxplot}\NormalTok{(}\FunctionTok{aes}\NormalTok{(}\AttributeTok{x =} \FunctionTok{factor}\NormalTok{(totalHijos), }\AttributeTok{y =}\NormalTok{ MntFruits))}
\NormalTok{gb3}\OtherTok{\textless{}{-}}\FunctionTok{ggplot}\NormalTok{(datos\_ML) }\SpecialCharTok{+} 
  \FunctionTok{geom\_boxplot}\NormalTok{(}\FunctionTok{aes}\NormalTok{(}\AttributeTok{x =} \FunctionTok{factor}\NormalTok{(totalHijos), }\AttributeTok{y =}\NormalTok{ MntMeatProducts))}
\NormalTok{gb4}\OtherTok{\textless{}{-}}\FunctionTok{ggplot}\NormalTok{(datos\_ML) }\SpecialCharTok{+} 
  \FunctionTok{geom\_boxplot}\NormalTok{(}\FunctionTok{aes}\NormalTok{(}\AttributeTok{x =} \FunctionTok{factor}\NormalTok{(totalHijos), }\AttributeTok{y =}\NormalTok{ MntFishProducts))}
\NormalTok{gb5}\OtherTok{\textless{}{-}}\FunctionTok{ggplot}\NormalTok{(datos\_ML) }\SpecialCharTok{+} 
  \FunctionTok{geom\_boxplot}\NormalTok{(}\FunctionTok{aes}\NormalTok{(}\AttributeTok{x =} \FunctionTok{factor}\NormalTok{(totalHijos), }\AttributeTok{y =}\NormalTok{ MntSweetProducts))}
\NormalTok{gb6}\OtherTok{\textless{}{-}}\FunctionTok{ggplot}\NormalTok{(datos\_ML) }\SpecialCharTok{+} 
  \FunctionTok{geom\_boxplot}\NormalTok{(}\FunctionTok{aes}\NormalTok{(}\AttributeTok{x =} \FunctionTok{factor}\NormalTok{(totalHijos), }\AttributeTok{y =}\NormalTok{ MntGoldProds))}
\NormalTok{gb7}\OtherTok{\textless{}{-}}\FunctionTok{ggplot}\NormalTok{(datos\_ML) }\SpecialCharTok{+} 
  \FunctionTok{geom\_boxplot}\NormalTok{(}\FunctionTok{aes}\NormalTok{(}\AttributeTok{x =} \FunctionTok{factor}\NormalTok{(totalHijos), }\AttributeTok{y =}\NormalTok{ Dinero\_Gastado))}
\NormalTok{gridExtra}\SpecialCharTok{::}\FunctionTok{grid.arrange}\NormalTok{(gb1,gb2,gb3,gb4, gb5, gb6, gb7)}
\end{Highlighting}
\end{Shaded}

\begin{figure}
\centering
\includegraphics{Proyecto_files/figure-latex/unnamed-chunk-19-1.pdf}
\caption{Gráfico relación nº hijos y gasto}
\end{figure}

Siguiendo con la relación anterior, observamos que a más hijos menor es
el gasto en todos los productos.

Veamos la relación entre la edad y los gastos de compras. Primero vamos
a ver la distribución de la edad.

\begin{Shaded}
\begin{Highlighting}[]
\FunctionTok{ggplot}\NormalTok{(datos) }\SpecialCharTok{+}
  \FunctionTok{geom\_histogram}\NormalTok{(}\FunctionTok{aes}\NormalTok{(}\AttributeTok{x =}\NormalTok{ edad, }\AttributeTok{y =}\FunctionTok{stat}\NormalTok{(density)), }\AttributeTok{bins =} \DecValTok{15}\NormalTok{, }\AttributeTok{fill =} \StringTok{"darkgreen"}\NormalTok{, }
                 \AttributeTok{color =} \StringTok{"black"}\NormalTok{) }\SpecialCharTok{+}
  \FunctionTok{geom\_density}\NormalTok{(}\FunctionTok{aes}\NormalTok{(}\AttributeTok{x =}\NormalTok{ edad), }\AttributeTok{color=}\StringTok{"red"}\NormalTok{, }\AttributeTok{size=}\FloatTok{1.5}\NormalTok{)}
\end{Highlighting}
\end{Shaded}

\begin{figure}
\centering
\includegraphics{Proyecto_files/figure-latex/unnamed-chunk-20-1.pdf}
\caption{Distribución edad}
\end{figure}

Como hemos comentado anteriormente existe un valor atípico dentro de
nuestra variable, por lo que al unicamente ser uno lo eliminaremos (ya
que no afectará a la información) y volvemos a examinar los datos.

\begin{Shaded}
\begin{Highlighting}[]
\NormalTok{datos }\OtherTok{\textless{}{-}}\NormalTok{ datos }\SpecialCharTok{\%\textgreater{}\%}
  \FunctionTok{filter}\NormalTok{(edad }\SpecialCharTok{\textless{}} \DecValTok{100}\NormalTok{)}
\FunctionTok{ggplot}\NormalTok{(datos) }\SpecialCharTok{+}
  \FunctionTok{geom\_histogram}\NormalTok{(}\FunctionTok{aes}\NormalTok{(}\AttributeTok{x =}\NormalTok{ edad, }\AttributeTok{y =}\FunctionTok{stat}\NormalTok{(density)), }\AttributeTok{bins =} \DecValTok{15}\NormalTok{, }\AttributeTok{fill =} \StringTok{"darkgreen"}\NormalTok{,}
                 \AttributeTok{color =}\StringTok{\textquotesingle{}black\textquotesingle{}}\NormalTok{) }\SpecialCharTok{+}
  \FunctionTok{geom\_density}\NormalTok{(}\FunctionTok{aes}\NormalTok{(}\AttributeTok{x =}\NormalTok{ edad), }\AttributeTok{color=}\StringTok{"red"}\NormalTok{, }\AttributeTok{size=}\FloatTok{1.5}\NormalTok{)}
\end{Highlighting}
\end{Shaded}

\begin{figure}
\centering
\includegraphics{Proyecto_files/figure-latex/unnamed-chunk-21-1.pdf}
\caption{Distribución edad retocado}
\end{figure}

Observamos que nuestros datos ya tienen una forma que es plausible, por
lo que podemos iniciar el análisis de los datos.

\begin{Shaded}
\begin{Highlighting}[]
\NormalTok{g1 }\OtherTok{\textless{}{-}}\FunctionTok{ggplot}\NormalTok{(datos) }\SpecialCharTok{+} 
  \FunctionTok{geom\_jitter}\NormalTok{(}\FunctionTok{aes}\NormalTok{(}\AttributeTok{x =}\NormalTok{ edad, }\AttributeTok{y =}\NormalTok{ MntWines))}
\NormalTok{g2}\OtherTok{\textless{}{-}}\FunctionTok{ggplot}\NormalTok{(datos) }\SpecialCharTok{+} 
  \FunctionTok{geom\_jitter}\NormalTok{(}\FunctionTok{aes}\NormalTok{(}\AttributeTok{x =}\NormalTok{ edad, }\AttributeTok{y =}\NormalTok{ MntFruits))}
\NormalTok{g3}\OtherTok{\textless{}{-}}\FunctionTok{ggplot}\NormalTok{(datos) }\SpecialCharTok{+} 
  \FunctionTok{geom\_jitter}\NormalTok{(}\FunctionTok{aes}\NormalTok{(}\AttributeTok{x =}\NormalTok{ edad, }\AttributeTok{y =}\NormalTok{ MntMeatProducts))}
\NormalTok{g4}\OtherTok{\textless{}{-}}\FunctionTok{ggplot}\NormalTok{(datos) }\SpecialCharTok{+} 
  \FunctionTok{geom\_jitter}\NormalTok{(}\FunctionTok{aes}\NormalTok{(}\AttributeTok{x =}\NormalTok{ edad, }\AttributeTok{y =}\NormalTok{ MntFishProducts))}
\NormalTok{g5}\OtherTok{\textless{}{-}}\FunctionTok{ggplot}\NormalTok{(datos) }\SpecialCharTok{+} 
  \FunctionTok{geom\_jitter}\NormalTok{(}\FunctionTok{aes}\NormalTok{(}\AttributeTok{x =}\NormalTok{ edad, }\AttributeTok{y =}\NormalTok{ MntSweetProducts))}
\NormalTok{g6}\OtherTok{\textless{}{-}}\FunctionTok{ggplot}\NormalTok{(datos) }\SpecialCharTok{+} 
  \FunctionTok{geom\_jitter}\NormalTok{(}\FunctionTok{aes}\NormalTok{(}\AttributeTok{x =}\NormalTok{ edad, }\AttributeTok{y =}\NormalTok{ MntGoldProds))}
\NormalTok{g7}\OtherTok{\textless{}{-}}\FunctionTok{ggplot}\NormalTok{(datos) }\SpecialCharTok{+} 
  \FunctionTok{geom\_jitter}\NormalTok{(}\FunctionTok{aes}\NormalTok{(}\AttributeTok{x =}\NormalTok{ edad, }\AttributeTok{y =}\NormalTok{ Dinero\_Gastado))}
\NormalTok{gridExtra}\SpecialCharTok{::}\FunctionTok{grid.arrange}\NormalTok{(g1,g2,g3,g4, g5, g6, g7)}
\end{Highlighting}
\end{Shaded}

\begin{figure}
\centering
\includegraphics{Proyecto_files/figure-latex/unnamed-chunk-22-1.pdf}
\caption{Gráfico edad y gasto por productos}
\end{figure}

En una primera observación podemos ver que el gasto en vino es mucho
mayor que en cualquiera de las otras secciones y para cualquier edad.

Por otra parte, no parece que existan patrones muy claros en ninguna de
las variables. Sin embargo, podríamos decir que las personas de mayor
edad gastan más dinero en vino y eso probablemente repercute en que
gasten más dinero en general.

Realizamos un último estudio en función del nivel de estudios y el
estado sentimental. Pero primero vamos a ver cuantos datos hay de cada
tipo.

\begin{Shaded}
\begin{Highlighting}[]
\FunctionTok{table}\NormalTok{(datos}\SpecialCharTok{$}\NormalTok{Education, datos}\SpecialCharTok{$}\NormalTok{Marital\_Status)}
\end{Highlighting}
\end{Shaded}

\begin{verbatim}
##             
##              Single Not single
##   Basic          19         35
##   Graduation    364        750
##   Master        112        253
##   PhD           148        332
\end{verbatim}

\begin{Shaded}
\begin{Highlighting}[]
\FunctionTok{ggplot}\NormalTok{(datos) }\SpecialCharTok{+}
  \FunctionTok{geom\_bar}\NormalTok{(}\FunctionTok{aes}\NormalTok{(}\AttributeTok{x =}\NormalTok{ Education, }\AttributeTok{fill =}\NormalTok{ Marital\_Status), }\AttributeTok{position =} \StringTok{"dodge"}\NormalTok{)}
\end{Highlighting}
\end{Shaded}

\begin{figure}
\centering
\includegraphics{Proyecto_files/figure-latex/unnamed-chunk-24-1.pdf}
\caption{Relación educación y estado civil}
\end{figure}

Anteriormente hemos visto de forma general que el número de clientes con
pareja son el doble de los sin pareja, pero ahora vemos esta relación se
cumple además, para todos los grupos de niveles de estudio

Veamos la relación entre el salario y el nivel de estudios:

\begin{Shaded}
\begin{Highlighting}[]
\FunctionTok{ggplot}\NormalTok{(datos) }\SpecialCharTok{+}
  \FunctionTok{geom\_boxplot}\NormalTok{(}\FunctionTok{aes}\NormalTok{(}\AttributeTok{x =} \FunctionTok{factor}\NormalTok{(Education), }\AttributeTok{y =}\NormalTok{ Income))}
\end{Highlighting}
\end{Shaded}

\begin{figure}
\centering
\includegraphics{Proyecto_files/figure-latex/unnamed-chunk-25-1.pdf}
\caption{Relación salario y nivel de estudios}
\end{figure}

Vemos que hay un outlier que no nos permite ver correctamente los
gráficos. Por este motivo lo quitamos.

\begin{Shaded}
\begin{Highlighting}[]
\NormalTok{aux }\OtherTok{\textless{}{-}}\NormalTok{ datos }\SpecialCharTok{\%\textgreater{}\%}
  \FunctionTok{filter}\NormalTok{(Income }\SpecialCharTok{\textless{}} \DecValTok{500000}\NormalTok{)}
\FunctionTok{ggplot}\NormalTok{(aux) }\SpecialCharTok{+}
  \FunctionTok{geom\_boxplot}\NormalTok{(}\FunctionTok{aes}\NormalTok{(}\AttributeTok{x =} \FunctionTok{factor}\NormalTok{(Education), }\AttributeTok{y =}\NormalTok{ Income))}
\end{Highlighting}
\end{Shaded}

\begin{figure}
\centering
\includegraphics{Proyecto_files/figure-latex/unnamed-chunk-26-1.pdf}
\caption{Modificación outlier}
\end{figure}

Se puede observar cláramente que las personas con un nivel de estudio
``Basic'' ganan mucho menos que cualquiera de los otros 3 grupos. Otra
cosa que hay que tener en cuenta es que entre los 3 grupos restantes no
existen deiferencias significativas.

Por último vemos la relación entre el gasto y el nivel de estudios y la
situación sentimental.

\begin{Shaded}
\begin{Highlighting}[]
\NormalTok{g1 }\OtherTok{\textless{}{-}} \FunctionTok{ggplot}\NormalTok{(datos) }\SpecialCharTok{+}
  \FunctionTok{geom\_col}\NormalTok{(}\FunctionTok{aes}\NormalTok{(}\AttributeTok{x =}\NormalTok{ Education, }\AttributeTok{y =}\NormalTok{ MntWines, }\AttributeTok{fill =}\NormalTok{ Marital\_Status),}\AttributeTok{position =} \StringTok{"dodge"}\NormalTok{)}
\NormalTok{g2}\OtherTok{\textless{}{-}}\FunctionTok{ggplot}\NormalTok{(datos) }\SpecialCharTok{+} 
  \FunctionTok{geom\_col}\NormalTok{(}\FunctionTok{aes}\NormalTok{(}\AttributeTok{x =}\NormalTok{ Education, }\AttributeTok{y =}\NormalTok{ MntFruits, }\AttributeTok{fill =}\NormalTok{ Marital\_Status),}\AttributeTok{position =} \StringTok{"dodge"}\NormalTok{)}
\NormalTok{g3}\OtherTok{\textless{}{-}}\FunctionTok{ggplot}\NormalTok{(datos) }\SpecialCharTok{+} 
  \FunctionTok{geom\_col}\NormalTok{(}\FunctionTok{aes}\NormalTok{(}\AttributeTok{x =}\NormalTok{ Education, }\AttributeTok{y =}\NormalTok{ MntMeatProducts, }\AttributeTok{fill =}\NormalTok{ Marital\_Status),}\AttributeTok{position =} \StringTok{"dodge"}\NormalTok{)}
\NormalTok{g4}\OtherTok{\textless{}{-}}\FunctionTok{ggplot}\NormalTok{(datos) }\SpecialCharTok{+} 
  \FunctionTok{geom\_col}\NormalTok{(}\FunctionTok{aes}\NormalTok{(}\AttributeTok{x =}\NormalTok{ Education, }\AttributeTok{y =}\NormalTok{ MntFishProducts, }\AttributeTok{fill =}\NormalTok{ Marital\_Status),}\AttributeTok{position =} \StringTok{"dodge"}\NormalTok{)}
\NormalTok{g5}\OtherTok{\textless{}{-}}\FunctionTok{ggplot}\NormalTok{(datos) }\SpecialCharTok{+} 
  \FunctionTok{geom\_col}\NormalTok{(}\FunctionTok{aes}\NormalTok{(}\AttributeTok{x =}\NormalTok{ Education, }\AttributeTok{y =}\NormalTok{ MntSweetProducts, }\AttributeTok{fill =}\NormalTok{ Marital\_Status),}\AttributeTok{position =} \StringTok{"dodge"}\NormalTok{)}
\NormalTok{g6}\OtherTok{\textless{}{-}}\FunctionTok{ggplot}\NormalTok{(datos) }\SpecialCharTok{+} 
  \FunctionTok{geom\_col}\NormalTok{(}\FunctionTok{aes}\NormalTok{(}\AttributeTok{x =}\NormalTok{ Education, }\AttributeTok{y =}\NormalTok{ MntGoldProds, }\AttributeTok{fill =}\NormalTok{ Marital\_Status),}\AttributeTok{position =} \StringTok{"dodge"}\NormalTok{)}
\NormalTok{g7}\OtherTok{\textless{}{-}}\FunctionTok{ggplot}\NormalTok{(datos) }\SpecialCharTok{+} 
  \FunctionTok{geom\_col}\NormalTok{(}\FunctionTok{aes}\NormalTok{(}\AttributeTok{x =}\NormalTok{ Education, }\AttributeTok{y =}\NormalTok{ Dinero\_Gastado, }\AttributeTok{fill =}\NormalTok{ Marital\_Status),}\AttributeTok{position =} \StringTok{"dodge"}\NormalTok{)}
\NormalTok{gridExtra}\SpecialCharTok{::}\FunctionTok{grid.arrange}\NormalTok{(g1,g2,g3,g4, g5, g6, g7)}
\end{Highlighting}
\end{Shaded}

\begin{figure}
\centering
\includegraphics{Proyecto_files/figure-latex/unnamed-chunk-27-1.pdf}
\caption{Relación gasto y nivel de estudios}
\end{figure}

Estos gráficos reflejan lo visto anteriormente, ya que en todas las
secciones, los clientes con un nivel de estudio ``Basic'' gastan mucho
menos dinero que cualquiera de los otros 3 grupos, lo que concuerda con
que su salario sea menor.

\newpage

\hypertarget{anuxe1lisis-predictivo-y-analuxedtica-avanzada}{%
\section{5. Análisis Predictivo y Analítica
Avanzada}\label{anuxe1lisis-predictivo-y-analuxedtica-avanzada}}

En esta sección nuestro objetivo es usar técnicas de Machine Learning
para predecir tendencias y comportamientos sobre diferentes variables
que nos puedan resultar interesantes. Antes de iniciar cualquier estudio
lo que haremos será definir nuestros parámetros de control.

\hypertarget{ajuste-paruxe1metros-de-control}{%
\subsection{5.1. Ajuste parámetros de
control}\label{ajuste-paruxe1metros-de-control}}

Lo que haremos para para tener siempre un mismo resultado es usar una
semilla, en nuestro caso será 2021. Por otro lado para el tema de
control usaremos el método de cross-validation con un fold de 10.

\begin{Shaded}
\begin{Highlighting}[]
\NormalTok{ctrl }\OtherTok{\textless{}{-}} \FunctionTok{trainControl}\NormalTok{(}\AttributeTok{method =} \StringTok{"cv"}\NormalTok{,}\AttributeTok{number =} \DecValTok{10}\NormalTok{,}\AttributeTok{summaryFunction =}\NormalTok{ defaultSummary,}
                     \AttributeTok{classProbs =} \ConstantTok{TRUE}\NormalTok{)}
\end{Highlighting}
\end{Shaded}

\hypertarget{anuxe1lisis-sobre-variable-complain}{%
\subsection{5.2. Análisis sobre variable
Complain}\label{anuxe1lisis-sobre-variable-complain}}

Otro análisis que se puede realizar mediante técnicas de machine
learning es encontrar aquellas variables que son claves a la hora de
detectar de forma anticipada que clientes se pueden quejar. Para ello
trabajaremos y realizaremos diferentes modelos de clasificación.

Lo primero antes de iniciar cualquier modelo, será ver como está
distribuida la variable complain y también cambiarla para poder trabajar
con ella:

\begin{Shaded}
\begin{Highlighting}[]
\NormalTok{datos}\SpecialCharTok{$}\NormalTok{Complain }\OtherTok{\textless{}{-}} \FunctionTok{ifelse}\NormalTok{(datos}\SpecialCharTok{$}\NormalTok{Complain }\SpecialCharTok{==} \DecValTok{1}\NormalTok{, }\StringTok{\textquotesingle{}Yes\textquotesingle{}}\NormalTok{,}\StringTok{\textquotesingle{}No\textquotesingle{}}\NormalTok{)}
\FunctionTok{table}\NormalTok{(datos}\SpecialCharTok{$}\NormalTok{Complain)}
\end{Highlighting}
\end{Shaded}

\begin{verbatim}
## 
##   No  Yes 
## 1996   17
\end{verbatim}

Tenemos que el dataset está totalmente desbalanceado, para evitar este
problema lo que haremos será rebalancear nuestros datos. Esto nos puede
generar alguna anomalía ya que estamos tratando los datos iniciales.

\begin{Shaded}
\begin{Highlighting}[]
\FunctionTok{set.seed}\NormalTok{(}\DecValTok{2021}\NormalTok{)}
\NormalTok{datos\_comp\_rebal }\OtherTok{\textless{}{-}} \FunctionTok{ovun.sample}\NormalTok{(Complain }\SpecialCharTok{\textasciitilde{}}\NormalTok{ ., }\AttributeTok{data =}\NormalTok{ datos, }\AttributeTok{method =} \StringTok{\textquotesingle{}both\textquotesingle{}}\NormalTok{, }
                                \AttributeTok{N =} \FunctionTok{table}\NormalTok{(datos}\SpecialCharTok{$}\NormalTok{Complain)[}\DecValTok{1}\NormalTok{]}\SpecialCharTok{*}\DecValTok{2}\NormalTok{)}\SpecialCharTok{$}\NormalTok{data}
\FunctionTok{table}\NormalTok{(datos\_comp\_rebal}\SpecialCharTok{$}\NormalTok{Complain)}
\end{Highlighting}
\end{Shaded}

\begin{verbatim}
## 
##   No  Yes 
## 2027 1965
\end{verbatim}

\begin{Shaded}
\begin{Highlighting}[]
\NormalTok{datos\_comp\_rebal }\OtherTok{\textless{}{-}}\NormalTok{ datos\_comp\_rebal }\SpecialCharTok{\%\textgreater{}\%}
  \FunctionTok{select}\NormalTok{(}\SpecialCharTok{{-}}\NormalTok{ID)}
\NormalTok{datos\_comp\_rebal}\SpecialCharTok{$}\NormalTok{Complain }\OtherTok{\textless{}{-}} \FunctionTok{as.factor}\NormalTok{(datos\_comp\_rebal}\SpecialCharTok{$}\NormalTok{Complain)}
\end{Highlighting}
\end{Shaded}

Una vez que tenemos los datos rebalanceados podemos pasar al siguiente
paso, que consistirá en mirar si tenemos variables que tengan una alta
correlación:

\begin{Shaded}
\begin{Highlighting}[]
\NormalTok{catvars }\OtherTok{\textless{}{-}} \FunctionTok{sapply}\NormalTok{(datos\_comp\_rebal, class) }\SpecialCharTok{\%in\%} \FunctionTok{c}\NormalTok{(}\StringTok{"character"}\NormalTok{,}\StringTok{"factpr"}\NormalTok{)}
\NormalTok{numvars }\OtherTok{\textless{}{-}} \FunctionTok{sapply}\NormalTok{(datos\_comp\_rebal, class) }\SpecialCharTok{\%in\%} \FunctionTok{c}\NormalTok{(}\StringTok{"integer"}\NormalTok{,}\StringTok{"numeric"}\NormalTok{)}
\NormalTok{C }\OtherTok{\textless{}{-}} \FunctionTok{cor}\NormalTok{(datos\_comp\_rebal[,numvars])}
\FunctionTok{corrplot}\NormalTok{(C, }\AttributeTok{method =} \StringTok{"circle"}\NormalTok{)}
\end{Highlighting}
\end{Shaded}

\begin{figure}

{\centering \includegraphics{Proyecto_files/figure-latex/unnamed-chunk-31-1} 

}

\caption{Gráfico de correlación}\label{fig:unnamed-chunk-31}
\end{figure}

Vemos que muchas variables tienen una alta correlación entre si, por
ahora no haremos nada pero posteriormente veremos si es necesario
eliminar alguna o no.

Una vez hecho esto podemos pasar a la parte de modelos, lo primero que
haremos será dividir el conjunto de datos entre entrenamiento y test.

\begin{Shaded}
\begin{Highlighting}[]
\FunctionTok{set.seed}\NormalTok{(}\DecValTok{2021}\NormalTok{)}
\NormalTok{trainIndex2 }\OtherTok{\textless{}{-}} \FunctionTok{createDataPartition}\NormalTok{(datos\_comp\_rebal}\SpecialCharTok{$}\NormalTok{Complain, }\AttributeTok{p =} \FloatTok{0.8}\NormalTok{, }\AttributeTok{list =} \ConstantTok{FALSE}\NormalTok{, }\AttributeTok{times =} \DecValTok{1}\NormalTok{)}
\NormalTok{fTR2 }\OtherTok{\textless{}{-}}\NormalTok{ datos\_comp\_rebal[trainIndex2,]}
\NormalTok{fTS2 }\OtherTok{\textless{}{-}}\NormalTok{ datos\_comp\_rebal[}\SpecialCharTok{{-}}\NormalTok{trainIndex2,] }
\NormalTok{fTR2\_eval }\OtherTok{\textless{}{-}}\NormalTok{ fTR2}
\NormalTok{fTS2\_eval }\OtherTok{\textless{}{-}}\NormalTok{ fTS2}
\end{Highlighting}
\end{Shaded}

\newpage

Una vez definido tanto el conjunto de entrenamiento como el de test
realizaremos un modelo sencillo para ver cómo se comporta todo, este
será una regresión logística. A su vez también usaremos como método de
control un cross-validation con un fold de 10.

\begin{Shaded}
\begin{Highlighting}[]
\FunctionTok{set.seed}\NormalTok{(}\DecValTok{2021}\NormalTok{)}
\NormalTok{LogReg.fit }\OtherTok{\textless{}{-}} \FunctionTok{train}\NormalTok{(}\AttributeTok{form =}\NormalTok{ Complain }\SpecialCharTok{\textasciitilde{}}\NormalTok{ . , }\AttributeTok{data =}\NormalTok{ fTR2, }\AttributeTok{method =} \StringTok{"glm"}\NormalTok{, }
                    \AttributeTok{trControl =}\NormalTok{ ctrl, }\AttributeTok{metric =} \StringTok{"Accuracy"}\NormalTok{)  }
\NormalTok{LogReg.fit  }
\end{Highlighting}
\end{Shaded}

\begin{verbatim}
## Generalized Linear Model 
## 
## 3194 samples
##   16 predictor
##    2 classes: 'No', 'Yes' 
## 
## No pre-processing
## Resampling: Cross-Validated (10 fold) 
## Summary of sample sizes: 2875, 2874, 2875, 2875, 2875, 2874, ... 
## Resampling results:
## 
##   Accuracy   Kappa    
##   0.7069357  0.4140293
\end{verbatim}

Podemos observar que nuestro primer modelo, donde tenemos todas las
variables, nos devuelve un accuracy de 0.695. Lo cual no esta mal para
empezar, pero vayamos a lo que realmente nos interesa, que variables son
importantes.

\begin{Shaded}
\begin{Highlighting}[]
\FunctionTok{summary}\NormalTok{(LogReg.fit)}
\end{Highlighting}
\end{Shaded}

\begin{verbatim}
## 
## Call:
## NULL
## 
## Deviance Residuals: 
##      Min        1Q    Median        3Q       Max  
## -2.79747  -0.83460  -0.00053   0.83918   1.34729  
## 
## Coefficients: (1 not defined because of singularities)
##                              Estimate Std. Error z value Pr(>|z|)    
## (Intercept)                -1.495e+01  1.980e+02  -0.075 0.939825    
## EducationGraduation         1.637e+01  1.980e+02   0.083 0.934103    
## EducationMaster             1.567e+01  1.980e+02   0.079 0.936912    
## EducationPhD                1.380e+01  1.980e+02   0.070 0.944443    
## `Marital_StatusNot single` -7.128e-01  1.124e-01  -6.344 2.24e-10 ***
## Income                     -1.780e-06  2.161e-06  -0.824 0.410176    
## Dt_Customer.L              -1.058e+00  1.006e-01 -10.518  < 2e-16 ***
## Dt_Customer.Q               3.357e-01  8.705e-02   3.857 0.000115 ***
## MntWines                   -2.759e-03  3.899e-04  -7.077 1.47e-12 ***
## MntFruits                   1.757e-02  2.448e-03   7.178 7.06e-13 ***
## MntMeatProducts            -2.869e-03  5.185e-04  -5.533 3.15e-08 ***
## MntFishProducts            -1.932e-03  1.858e-03  -1.040 0.298366    
## MntSweetProducts           -2.358e-02  2.563e-03  -9.199  < 2e-16 ***
## MntGoldProds               -1.435e-02  1.409e-03 -10.183  < 2e-16 ***
## NumWebVisitsMonth          -2.455e-01  3.163e-02  -7.761 8.45e-15 ***
## Response1                   1.110e+00  1.802e-01   6.162 7.18e-10 ***
## edad                        2.517e-02  4.269e-03   5.895 3.75e-09 ***
## totalHijos                  1.233e-01  7.844e-02   1.572 0.116008    
## suma_compras                6.074e-02  1.505e-02   4.036 5.43e-05 ***
## Dinero_Gastado                     NA         NA      NA       NA    
## ---
## Signif. codes:  0 '***' 0.001 '**' 0.01 '*' 0.05 '.' 0.1 ' ' 1
## 
## (Dispersion parameter for binomial family taken to be 1)
## 
##     Null deviance: 4427.0  on 3193  degrees of freedom
## Residual deviance: 3266.7  on 3175  degrees of freedom
## AIC: 3304.7
## 
## Number of Fisher Scoring iterations: 14
\end{verbatim}

Vemos que las variables importantes en nuestro modelo son el estado
civil, la antigüedad del cliente, el consumo en diferentes productos,
las veces que visitan la web, si respondes a la ofertas, la edad y el
total de compras. Cabe destacar que este modelo solo detecta importancia
de variables lineales con el output.

Evaluamos nuestro modelo para obtener datos más claros.

\begin{Shaded}
\begin{Highlighting}[]
\FunctionTok{set.seed}\NormalTok{(}\DecValTok{2021}\NormalTok{)}
\NormalTok{fTR2\_eval}\SpecialCharTok{$}\NormalTok{LRprob }\OtherTok{\textless{}{-}} \FunctionTok{predict}\NormalTok{(LogReg.fit, }\AttributeTok{type=}\StringTok{"prob"}\NormalTok{, }\AttributeTok{newdata =}\NormalTok{ fTR2)}
\NormalTok{fTR2\_eval}\SpecialCharTok{$}\NormalTok{LRpred }\OtherTok{\textless{}{-}} \FunctionTok{predict}\NormalTok{(LogReg.fit, }\AttributeTok{type=}\StringTok{"raw"}\NormalTok{, }\AttributeTok{newdata =}\NormalTok{ fTR2)}
\NormalTok{fTS2\_eval}\SpecialCharTok{$}\NormalTok{LRprob }\OtherTok{\textless{}{-}} \FunctionTok{predict}\NormalTok{(LogReg.fit, }\AttributeTok{type=}\StringTok{"prob"}\NormalTok{, }\AttributeTok{newdata =}\NormalTok{ fTS2) }
\NormalTok{fTS2\_eval}\SpecialCharTok{$}\NormalTok{LRpred }\OtherTok{\textless{}{-}} \FunctionTok{predict}\NormalTok{(LogReg.fit, }\AttributeTok{type=}\StringTok{"raw"}\NormalTok{, }\AttributeTok{newdata =}\NormalTok{ fTS2)}
\end{Highlighting}
\end{Shaded}

Ahora implementaremos otro modelo, que será el de arbol de decisiones.
Este tiene la ventaja de que si nos dará las variables más importantes
incluso si tienen relación no lineal con el output.

\begin{Shaded}
\begin{Highlighting}[]
\FunctionTok{set.seed}\NormalTok{(}\DecValTok{2021}\NormalTok{)}
\NormalTok{tree2.fit }\OtherTok{\textless{}{-}} \FunctionTok{train}\NormalTok{(}\AttributeTok{x =}\NormalTok{ fTR2[,}\FunctionTok{c}\NormalTok{(}\FunctionTok{seq}\NormalTok{(}\DecValTok{1}\NormalTok{,}\DecValTok{11}\NormalTok{),}\FunctionTok{seq}\NormalTok{(}\DecValTok{13}\NormalTok{,}\DecValTok{17}\NormalTok{))],}\AttributeTok{y =}\NormalTok{ fTR2}\SpecialCharTok{$}\NormalTok{Complain,}\AttributeTok{method =}\StringTok{"rpart"}\NormalTok{,}
            \AttributeTok{control=}\FunctionTok{rpart.control}\NormalTok{(}\AttributeTok{minsplit=}\DecValTok{20}\NormalTok{,}\AttributeTok{minbucket =} \DecValTok{20}\NormalTok{), }\AttributeTok{parms =} \FunctionTok{list}\NormalTok{(}\AttributeTok{split =} \StringTok{"gini"}\NormalTok{),}
            \AttributeTok{tuneGrid =} \FunctionTok{data.frame}\NormalTok{(}\AttributeTok{cp =} \FunctionTok{seq}\NormalTok{(}\DecValTok{0}\NormalTok{,}\FloatTok{0.1}\NormalTok{,}\FloatTok{0.001}\NormalTok{)), }\AttributeTok{trControl =}\NormalTok{ ctrl,}\AttributeTok{metric =} \StringTok{"Accuracy"}\NormalTok{)}
\FunctionTok{ggplot}\NormalTok{(tree2.fit) }
\end{Highlighting}
\end{Shaded}

\begin{figure}

{\centering \includegraphics{Proyecto_files/figure-latex/unnamed-chunk-36-1} 

}

\caption{Gráfico de selección hiperparámetro}\label{fig:unnamed-chunk-36}
\end{figure}

Podemos observar que este modelo tiene un accuracy aproximado de 0.97
con un hiperparámetro c igual a 0. Elegimos \(c = 0\) ya que es el valor
que maximiza el accuracy.

A continuación veremos que variables son importantes, donde entrarán
también en juego aquellas variables las cuales tengan relaciones no
lineales con nuestro output.

\begin{Shaded}
\begin{Highlighting}[]
\FunctionTok{plot}\NormalTok{(}\FunctionTok{varImp}\NormalTok{(tree2.fit,}\AttributeTok{scale =} \ConstantTok{FALSE}\NormalTok{))}
\end{Highlighting}
\end{Shaded}

\begin{figure}

{\centering \includegraphics{Proyecto_files/figure-latex/unnamed-chunk-37-1} 

}

\caption{Importancia de variables}\label{fig:unnamed-chunk-37}
\end{figure}

\newpage

Obtenemos que las variables más importantes son el consumo de productos
``gourmet'', carne y vino así como también la edad, los ingresos y el
dinero gastado.

Por último haremos un modelo solo con las variables más importantes,
tanto aquellas que tienen una relación lineal como aquellas que son no
lineales. Para una mayor facilidad de comprensión del modelo y viendo
como ha salido el último, realizaremos un arbol de decisión.

\begin{Shaded}
\begin{Highlighting}[]
\FunctionTok{set.seed}\NormalTok{(}\DecValTok{2021}\NormalTok{)}
\NormalTok{tree2\_1.fit }\OtherTok{\textless{}{-}} \FunctionTok{train}\NormalTok{(}\AttributeTok{x =}\NormalTok{ fTR2[,}\FunctionTok{c}\NormalTok{(}\DecValTok{1}\NormalTok{,}\DecValTok{3}\NormalTok{,}\DecValTok{5}\NormalTok{,}\DecValTok{8}\NormalTok{,}\DecValTok{9}\NormalTok{,}\DecValTok{10}\NormalTok{,}\DecValTok{14}\NormalTok{,}\DecValTok{17}\NormalTok{)],}\AttributeTok{y =}\NormalTok{ fTR2}\SpecialCharTok{$}\NormalTok{Complain,}\AttributeTok{method =}\StringTok{"rpart"}\NormalTok{,}
            \AttributeTok{control=}\FunctionTok{rpart.control}\NormalTok{(}\AttributeTok{minsplit=}\DecValTok{20}\NormalTok{,}\AttributeTok{minbucket =} \DecValTok{20}\NormalTok{), }\AttributeTok{parms =} \FunctionTok{list}\NormalTok{(}\AttributeTok{split =} \StringTok{"gini"}\NormalTok{),}
            \AttributeTok{tuneGrid =} \FunctionTok{data.frame}\NormalTok{(}\AttributeTok{cp =} \FunctionTok{seq}\NormalTok{(}\DecValTok{0}\NormalTok{,}\FloatTok{0.1}\NormalTok{,}\FloatTok{0.001}\NormalTok{)), }\AttributeTok{trControl =}\NormalTok{ ctrl,}\AttributeTok{metric =} \StringTok{"Accuracy"}\NormalTok{)}
\FunctionTok{ggplot}\NormalTok{(tree2\_1.fit) }
\end{Highlighting}
\end{Shaded}

\begin{figure}

{\centering \includegraphics{Proyecto_files/figure-latex/unnamed-chunk-38-1} 

}

\caption{Gráfico de selección hiperparámetro}\label{fig:unnamed-chunk-38}
\end{figure}

\newpage

En este caso vemos que el accuracy no varía y que se vuelve a coger el
mismo valor de hiperparámetro.

Veamos como esta construido nuestro modelo de arbol de decisión:

\begin{Shaded}
\begin{Highlighting}[]
\FunctionTok{rpart.plot}\NormalTok{(tree2\_1.fit}\SpecialCharTok{$}\NormalTok{finalModel, }\AttributeTok{type =} \DecValTok{2}\NormalTok{, }\AttributeTok{fallen.leaves =} \ConstantTok{FALSE}\NormalTok{, }\AttributeTok{box.palette =} \StringTok{"Oranges"}\NormalTok{)}
\end{Highlighting}
\end{Shaded}

\begin{figure}

{\centering \includegraphics{Proyecto_files/figure-latex/unnamed-chunk-39-1} 

}

\caption{Estructura Arbol de Decisión}\label{fig:unnamed-chunk-39}
\end{figure}

Por último haremos como en el caso de la regresión logística, donde
representamos graficamente la predicción frente a los valores reales de
cada observación.

\begin{Shaded}
\begin{Highlighting}[]
\FunctionTok{set.seed}\NormalTok{(}\DecValTok{2021}\NormalTok{)}
\NormalTok{fTR2\_eval}\SpecialCharTok{$}\NormalTok{tree2\_1\_prob }\OtherTok{\textless{}{-}} \FunctionTok{predict}\NormalTok{(tree2\_1.fit, }\AttributeTok{type=}\StringTok{"prob"}\NormalTok{, }\AttributeTok{newdata =}\NormalTok{ fTR2) }
\NormalTok{fTR2\_eval}\SpecialCharTok{$}\NormalTok{tree2\_1\_pred }\OtherTok{\textless{}{-}} \FunctionTok{predict}\NormalTok{(tree2\_1.fit, }\AttributeTok{type=}\StringTok{"raw"}\NormalTok{, }\AttributeTok{newdata =}\NormalTok{ fTR2)}
\NormalTok{fTS2\_eval}\SpecialCharTok{$}\NormalTok{tree2\_1\_prob }\OtherTok{\textless{}{-}} \FunctionTok{predict}\NormalTok{(tree2\_1.fit, }\AttributeTok{type=}\StringTok{"prob"}\NormalTok{, }\AttributeTok{newdata =}\NormalTok{ fTS2) }
\NormalTok{fTS2\_eval}\SpecialCharTok{$}\NormalTok{tree2\_1\_pred }\OtherTok{\textless{}{-}} \FunctionTok{predict}\NormalTok{(tree2\_1.fit, }\AttributeTok{type=}\StringTok{"raw"}\NormalTok{, }\AttributeTok{newdata =}\NormalTok{ fTS2)}
\end{Highlighting}
\end{Shaded}

\begin{Shaded}
\begin{Highlighting}[]
\FunctionTok{Plot2DClass}\NormalTok{(fTR2[,}\FunctionTok{c}\NormalTok{(}\DecValTok{1}\NormalTok{,}\DecValTok{3}\NormalTok{,}\DecValTok{5}\NormalTok{,}\DecValTok{8}\NormalTok{,}\DecValTok{9}\NormalTok{,}\DecValTok{10}\NormalTok{,}\DecValTok{14}\NormalTok{,}\DecValTok{17}\NormalTok{)],fTR2}\SpecialCharTok{$}\NormalTok{Complain,tree2\_1.fit,}\AttributeTok{var1 =} \StringTok{"edad"}\NormalTok{, }
            \AttributeTok{var2 =} \StringTok{"Dinero\_Gastado"}\NormalTok{, }\AttributeTok{selClass =} \StringTok{"Yes"}\NormalTok{)}
\end{Highlighting}
\end{Shaded}

\begin{figure}
\centering
\includegraphics{Proyecto_files/figure-latex/unnamed-chunk-41-1.pdf}
\caption{Gráfico de predicción}
\end{figure}

\newpage

A su vez, para ver que tal trabaja tanto en training como en test
sacaremos ambas matrices de confusión:

\begin{Shaded}
\begin{Highlighting}[]
\FunctionTok{confusionMatrix}\NormalTok{(}\AttributeTok{data =}\NormalTok{ fTR2\_eval}\SpecialCharTok{$}\NormalTok{tree2\_1\_pred,}\AttributeTok{reference =}\NormalTok{ fTR2\_eval}\SpecialCharTok{$}\NormalTok{Complain,}\AttributeTok{positive =} \StringTok{"Yes"}\NormalTok{)}
\end{Highlighting}
\end{Shaded}

\begin{verbatim}
## Confusion Matrix and Statistics
## 
##           Reference
## Prediction   No  Yes
##        No  1580    0
##        Yes   42 1572
##                                           
##                Accuracy : 0.9869          
##                  95% CI : (0.9823, 0.9905)
##     No Information Rate : 0.5078          
##     P-Value [Acc > NIR] : < 2.2e-16       
##                                           
##                   Kappa : 0.9737          
##                                           
##  Mcnemar's Test P-Value : 2.509e-10       
##                                           
##             Sensitivity : 1.0000          
##             Specificity : 0.9741          
##          Pos Pred Value : 0.9740          
##          Neg Pred Value : 1.0000          
##              Prevalence : 0.4922          
##          Detection Rate : 0.4922          
##    Detection Prevalence : 0.5053          
##       Balanced Accuracy : 0.9871          
##                                           
##        'Positive' Class : Yes             
## 
\end{verbatim}

\begin{Shaded}
\begin{Highlighting}[]
\FunctionTok{set.seed}\NormalTok{(}\DecValTok{2021}\NormalTok{)}
\FunctionTok{confusionMatrix}\NormalTok{(}\AttributeTok{data =}\NormalTok{ fTS2\_eval}\SpecialCharTok{$}\NormalTok{tree2\_1\_pred,}\AttributeTok{reference =}\NormalTok{ fTS2\_eval}\SpecialCharTok{$}\NormalTok{Complain,}\AttributeTok{positive =} \StringTok{"Yes"}\NormalTok{)}
\end{Highlighting}
\end{Shaded}

\begin{verbatim}
## Confusion Matrix and Statistics
## 
##           Reference
## Prediction  No Yes
##        No  395   0
##        Yes  10 393
##                                          
##                Accuracy : 0.9875         
##                  95% CI : (0.9771, 0.994)
##     No Information Rate : 0.5075         
##     P-Value [Acc > NIR] : < 2.2e-16      
##                                          
##                   Kappa : 0.9749         
##                                          
##  Mcnemar's Test P-Value : 0.004427       
##                                          
##             Sensitivity : 1.0000         
##             Specificity : 0.9753         
##          Pos Pred Value : 0.9752         
##          Neg Pred Value : 1.0000         
##              Prevalence : 0.4925         
##          Detection Rate : 0.4925         
##    Detection Prevalence : 0.5050         
##       Balanced Accuracy : 0.9877         
##                                          
##        'Positive' Class : Yes            
## 
\end{verbatim}

Por tanto, podemos afirmar que las variables que más importancia tienen
para detectar que un cliente se va a quejar en un futuro son el gasto en
productos ``gourmet'' y vino así como la edad, los ingresos y el dinero
gastado. Esto se puede ver en la siguiente imagen:

\begin{Shaded}
\begin{Highlighting}[]
\FunctionTok{plot}\NormalTok{(}\FunctionTok{varImp}\NormalTok{(tree2\_1.fit,}\AttributeTok{scale =} \ConstantTok{FALSE}\NormalTok{))}
\end{Highlighting}
\end{Shaded}

\begin{figure}
\centering
\includegraphics{Proyecto_files/figure-latex/unnamed-chunk-44-1.pdf}
\caption{Gráfico importancia varibales}
\end{figure}

\newpage
\tableoffigures
\newpage

\end{document}
